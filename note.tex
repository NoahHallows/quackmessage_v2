\documentclass[a4paper,12pt]{article}

\usepackage{lstlstings}
\usepackage{xcolor}
\usepackage{booktabs}
\usepackage{enumitem}
\usepackage[colorlinks=true, linkcolor=blue]{hyperref}

\title{Quackmessage notes}
\author{Noah Hallows}
\date{December 2025}

\begin{document}

\section{Preliminaries}
    VXEdDSA extends XEdDSA to make it a verifiable random function.

    \subsection{Notation}

        Multiplication of intergers \(a\), \(b\) modulo prime \(p\) is written as \(a
        \cdot b \mod{p}\)
        Division is calulated as \(a \cdot b^{-1} \mod{p}\)
        \(inv(a) \mod{p}\) returns \(a^{-1} \mod{p}\) when \(a \neq 0\) and \(0\)
        otherwise. This may be calulated as \(inv(a) = a^{p-2} \mod{p}\)
        Integer variables are lowercasem, points and other variables are uppercase.
        Integer constants are also lowercase except the Montgomery curve constant.
        Byte sequences are in bold. A bold integer or elliptic curve point represents a fixed-length byte sequence encoding the value.

    An elliptic curve is a plane curve defined by the equation \(y^{2} = f(x)\),
    where \(f(x)\) is a cubic polynomial with no repeated roots.

    \subsection{Eliptic curve parameters}
        \begin{tabularx}{c|c}
            \textbf{Name} & \textbf{Definition} \\
            \hline
            \(B\) & Base point \\
            \(I\) & Identity point \\
            \(p\) & Field prime \\
            \(q\) & Order of base point (prime; \(q < p\), \(q \cdot B = I\)) \\
            \(c\) & Cofactor \\
            \(d\) & Twisted Edwards curve constant \\
            \(A\) & Montgomery curve constant \\
            \(n\) & Nonsquare integer modulo p \\
            \(|p|\) & \(celi(\log_{2}(p))\) \\
            \(|q|\) & \(cali(\log_{2}(q))\) \\

\end{document}
